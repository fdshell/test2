%! Author = francois
%! Date = 17/03/2022

% Preamble
\documentclass[11pt]{article}

% Packages
\usepackage{amsmath}
\usepackage{url}
\usepackage[french]{babel}
\usepackage[utf8]{inputenc}
% Document
\begin{document}

    Théorème de Mimi~\cite{mimi1}:

$ ronron + croquettes = sieste $

    Second théorème de Mimi:

Dans n'importe quelle théorie récursivement axiomatisable, cohérente et capable de « formaliser l'arithmétique »,
on peut construire un énoncé arithmétique qui ne peut être ni démontré ni réfuté dans cette théorie.


    \bibliography{main}
    \bibliographystyle{plain}

\end{document}
